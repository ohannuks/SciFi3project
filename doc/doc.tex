\documentclass[titlepage, 11pt]{article}
\usepackage[english]{babel}           % LANGEUAGE, UTF-8
\usepackage[utf8]{inputenc}
\usepackage[T1]{fontenc}
\usepackage{url}                      % Url
\usepackage{listing}                  % Code block library
\usepackage{algpseudocode}            % Pseudocode library
\usepackage{algorithm}                % Algorithms
\usepackage{amsmath}                  % ALL MATH FEATURES
\usepackage{graphicx}                 % GRAPHICS                  
\usepackage{float}                    % IMPROVED FIGURE POSITIONING
\usepackage[labelsep=period]{caption} % IMPROVED CAPTIONS
\usepackage{url}                      % WRITE URLS
\usepackage{tikz}                     % MAKE TIKZ VECTOR GRAPHICS
\usepackage{tocloft}                  % CONTROL TABLE OF CONTENTS
\usepackage[authoryear,round]{natbib} % REFERENCES
\usepackage{apalike}

\renewcommand{\familydefault}{lmr}    % SET DOCUMENT FONT
\makeatletter                         % SET NICE SECTION HEADING STYLES
\renewcommand{\section}{\@startsection
        {section}
        {2}
        {0mm}
        {1.2\baselineskip}
        {\baselineskip}
        {\centering\normalsize}}
\renewcommand{\subsection}{\@startsection
        {subsection}
        {2}
        {0mm}
        {.8\baselineskip}
        {.5\baselineskip}
        {\bfseries\normalsize}}
\renewcommand{\subsubsection}{\@startsection
        {subsubsection}
        {2}
        {0mm}
        {.5\baselineskip}
        {0mm}
        {\it\bfseries\normalsize}}
\makeatother

\begin{document}
%\section*{REFERENCES}
%\bibliography{NameOfReferenceFile}{}

%\lstset{language=Pascal}

%\bibliographystyle{plainnat}

\section*{DESCRIPTION OF SIMULATION}

This is the description of our simulation. We use particles to model dark
matter in an exapanding universe.

\section*{PLAN FOR POISSON SOLVER}

\subsection*{Equations to Solve}
The first step is to solve the gravitational potential $\phi$ in a given region 
from the Poisson's equation for gravity at time $t_0$
\begin{equation}
\nabla^2 \phi = 4\pi G\rho. 
\end{equation}
For this task we are given some initial mass density $\rho$, determined by a 
distribution of massive particles, and some boundary conditions.

Once a solution for the potential has been obtained, the corresponding force
field can be solved from 
\begin{equation}
\mathbf{F} = -\nabla \phi.
\end{equation}

Given the position and velocity for a particle at $t_0$, we can can solve a 
new position and velocity for the particle at time $t$ from Newton's second law
\begin{align}
\mathbf{v}(t) &= \int_{t_0}^{t}\frac{\mathbf{F}}{m}dt' + \mathbf{v}(t_0)\\
\mathbf{x}(t) &= \int_{t_0}^{t}\mathbf{v}(t')dt' + \mathbf{x}(t_0).
\end{align}
Here it is assumed that the time step is so small that the force field remains 
unchanged.

After calculating the new particle positions a new mass density can be
obtained and inserted back to the Poisson's equation.



\subsection*{Algorithm}

The \textbf{Poisson equation} can be solved with The Successive Over Relaxation (SOR) 
method documented in \url{http://www.physics.buffalo.edu/phy410-505/2011/topic3/app1/index.html}. 

For the particles, we update particle speed with simplest box-time integration method.

So in total, the main algorithm will have three parts; calculating potential $\phi$ 
from density $\rho$, updating particles using $\phi$, then re-calculating density 
$\rho$.

Overview of the algorithm:

\begin{algorithm}
 \label{alg:main}
 \caption{Main program}
 \begin{algorithmic}
  \State Set up boundary conditions
  \State Set initial $\rho$
  \State Set initial guess for $\phi$
  \Loop
   \State Calculate $\phi \gets \phi_{next}$
   \For{ Every particle $p$ }
    \State Calculate new particle velocity and position for $p$
   \EndFor
   \State Calculate new $\rho$
   \State $t \gets t+dt$
  \EndLoop
 \end{algorithmic}

\end{algorithm}


\subsubsection*{Poisson equation.}\ For the Poisson problem, we use the SOR algorithm.

Pseudocode: 

\url{http://openi.nlm.nih.gov/detailedresult.php?img=2234413_1743-0003-4-46-16&req=4}

\url{http://www.physics.buffalo.edu/phy410-505/2011/topic3/app1/index.html}

\url{http://cs.ucsb.edu/~koc/docs/c13.pdf}

\begin{algorithm}
 \label{pseudo:SOR}
 \caption{SOR pseudocode}
\begin{algorithmic}[1]
 \Function{SORAlgorithm}{$grid$, tolerance}
  \For{ $n = 0,1,..n_{max_iterations}$ }
   \State Error $\sigma \gets 0$
   \For{ Every grid point $\phi_{i,j,k}$ }
    \State $\phi_{i,j,k}^{(n+1)} \gets (1-\omega)\phi_{i,j,k}^{(n)} + \frac{\omega}{6} (\phi_{i+1,j,k}^{(n)} + \phi_{i-1,j,k}^{(n)} + \phi_{i,j+1,k}^{(n)} + \phi_{i,j-1,k}^{(n)} + \phi_{i,j,k+1}^{(n)} + \phi_{i,j,k-1}^{(n)} + h^3\rho_{i,j,k})$
    \State Update $\sigma$
   \EndFor
   \State If $\sigma \leq tolerance$, break
  \EndFor
  \State \Return $grid$
 \EndFunction
\end{algorithmic}
\end{algorithm}



\end{document}






