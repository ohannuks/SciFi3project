\documentclass[titlepage, 11pt]{article}
\usepackage[english]{babel}           % LANGEUAGE, UTF-8
\usepackage[utf8]{inputenc}
\usepackage[T1]{fontenc}
\usepackage{amsmath}                  % ALL MATH FEATURES
\usepackage{graphicx}                 % GRAPHICS                  
\usepackage{float}                    % IMPROVED FIGURE POSITIONING
\usepackage[labelsep=period]{caption} % IMPROVED CAPTIONS
\usepackage{url}                      % WRITE URLS
\usepackage{tikz}                     % MAKE TIKZ VECTOR GRAPHICS
\usepackage{tocloft}                  % CONTROL TABLE OF CONTENTS
\usepackage[authoryear,round]{natbib} % REFERENCES
\usepackage{apalike}

\renewcommand{\familydefault}{lmr}    % SET DOCUMENT FONT
\makeatletter                         % SET NICE SECTION HEADING STYLES
\renewcommand{\section}{\@startsection
        {section}
        {2}
        {0mm}
        {1.2\baselineskip}
        {\baselineskip}
        {\centering\normalsize}}
\renewcommand{\subsection}{\@startsection
        {subsection}
        {2}
        {0mm}
        {.8\baselineskip}
        {.5\baselineskip}
        {\bfseries\normalsize}}
\renewcommand{\subsubsection}{\@startsection
        {subsubsection}
        {2}
        {0mm}
        {.5\baselineskip}
        {0mm}
        {\it\bfseries\normalsize}}
\makeatother

\begin{document}
%\section*{REFERENCES}
%\bibliography{NameOfReferenceFile}{}
%\bibliographystyle{plainnat}

\section*{DESCRIPTION OF SIMULATION}

This is the description of our simulation. We use particles to model dark
matter in an exapanding universe.

\section*{PLAN FOR POISSON SOLVER}

\subsection*{Equations to Solve}
The first step is to solve the gravitational potential $\phi$ in a given region 
from the Poisson's equation for gravity at time $t_0$
\begin{equation}
\nabla^2 \phi = 4\pi G\rho. 
\end{equation}
For this task we are given some initial mass density $\rho$, determined by a 
distribution of massive particles, and some boundary conditions.

Once a solution for the potential has been obtained, the corresponding force
field can be solved from 
\begin{equation}
\mathbf{F} = -\nabla \phi.
\end{equation}

Given the position and velocity for a particle at $t_0$, we can can solve a 
new position and velocity for the particle at time $t$ from Newton's second law
\begin{align}
\mathbf{v}(t) &= \int_{t_0}^{t}\frac{\mathbf{F}}{m}dt' + \mathbf{v}(t_0)\\
\mathbf{x}(t) &= \int_{t_0}^{t}\mathbf{v}(t')dt' + \mathbf{x}(t_0).
\end{align}
Here it is assumed that the time step is so small that the force field remains 
unchanged.

After calculating the new particle positions a new mass density can be
obtained and inserted back to the Poisson's equation.




\subsection*{Algorithm}

We use the SOR algorithm

The SOR algorithm is an algorithm that uses an iterative method for solving the poisson equation. It is more efficient than for instance 

\end{document}
